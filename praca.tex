\documentclass[pl,12pt]{aghdpl}
% \documentclass[en,11pt]{aghdpl}  % praca w języku angielskim

% Lista wszystkich języków stanowiących języki pozycji bibliograficznych użytych w pracy.
% (Zgodnie z zasadami tworzenia bibliografii każda pozycja powinna zostać utworzona zgodnie z zasadami języka, w którym dana publikacja została napisana.)
\usepackage[english,polish]{babel}

% Użyj polskiego łamania wyrazów (zamiast domyślnego angielskiego).
\usepackage{polski}

\usepackage[utf8]{inputenc}

% Załączniki

\usepackage[toc, page]{appendix}
\renewcommand\appendixpagename{Załączniki}
\renewcommand\appendixtocname{Załączniki}

% dodatkowe pakiety

\usepackage{mathtools}
\usepackage{amsfonts}
\usepackage{amsmath}
\usepackage{amsthm}
\usepackage{float}% do umieszczenia floatów [H]
\usepackage{enumitem}
\setlist{nosep} % or \setlist{noitemsep} to leave space around whole list
\usepackage[bookmarks,hidelinks]{hyperref}

% Środowisko float do kodu źródłowego \begin{program}

\floatstyle{plaintop}
\ifcsname{chapter}\endcsname%
    \newfloat{program}{!tbh}{lop}[chapter]
\else%
    \newfloat{program}{!tbh}{lop}
\fi
\floatname{program}{Kod źr.}

% Kod poniżej powoduje, że floaty nie wylatują poza granice sekcji

\usepackage{placeins}

\ifcsname{chapter}\endcsname%
    \let\Oldchapter\chapter%
    \renewcommand{\chapter}{\FloatBarrier\Oldchapter}
\fi

\let\Oldsection\section%
\renewcommand{\section}{\FloatBarrier\Oldsection}

\let\Oldsubsection\subsection%
\renewcommand{\subsection}{\FloatBarrier\Oldsubsection}

\let\Oldsubsubsection\subsubsection%
\renewcommand{\subsubsection}{\FloatBarrier\Oldsubsubsection}

% --- < bibliografia > ---


\usepackage[
style=numeric,
sorting=none,
%
% Zastosuj styl wpisu bibliograficznego właściwy językowi publikacji.
language=autobib,
autolang=other,
% Zapisuj datę dostępu do strony WWW w formacie RRRR-MM-DD.
urldate=iso,
seconds=true,
% Nie dodawaj numerów stron, na których występuje cytowanie.
backref=false,
% Podawaj ISBN.
isbn=true,
% Nie podawaj URL-i, o ile nie jest to konieczne.
url=false,
%
% Ustawienia związane z polskimi normami dla bibliografii.
maxbibnames=3,
% Jeżeli używamy Bibera:
backend=biber
]{biblatex}

\usepackage{csquotes}
% Ponieważ `csquotes` nie posiada polskiego stylu, można skorzystać z mocno zbliżonego stylu chorwackiego.
\DeclareQuoteAlias{croatian}{polish}

\addbibresource{bibliografia.bib}

% Przecinki zamiast kropek do oddzielenia pól wpisu bibliograficznego
% i dwukropek po nazwisku autora, bez kropki na końcu
\AtBeginBibliography{
    \renewcommand\labelnamepunct{:\space}
    \renewcommand\newunitpunct{\addcomma\space}
    \renewcommand{\finentrypunct}{}
    
    \renewcommand{\bibopenparen}{\addcomma\addspace}
    \renewcommand{\bibcloseparen}{\addspace}
}

% Nie wyświetlaj wybranych pól.
%\AtEveryBibitem{\clearfield{note}}


% ------------------------
% --- < listingi > ---

% Użyj czcionki kroju Times.
\usepackage{newtxtext}
\usepackage{newtxmath}

\usepackage{listings}
\lstset{language=TeX}

\lstset{%
        literate={ą}{{\k{a}}}1
           {ć}{{\'c}}1
           {ę}{{\k{e}}}1
           {ó}{{\'o}}1
           {ń}{{\'n}}1
           {ł}{{\l{}}}1
           {ś}{{\'s}}1
           {ź}{{\'z}}1
           {ż}{{\.z}}1
           {Ą}{{\k{A}}}1
           {Ć}{{\'C}}1
           {Ę}{{\k{E}}}1
           {Ó}{{\'O}}1
           {Ń}{{\'N}}1
           {Ł}{{\L{}}}1
           {Ś}{{\'S}}1
           {Ź}{{\'Z}}1
           {Ż}{{\.Z}}1
}

% Ustawienia pakietu lstlisting do umieszczania kodu

\usepackage{color}

\definecolor{mygreen}{rgb}{0,0.6,0}
\definecolor{mygray}{rgb}{0.5,0.5,0.5}
\definecolor{mymauve}{rgb}{0.58,0,0.82}

\lstset{%
  backgroundcolor=\color{white},     % choose the background color
  basicstyle=\ttfamily\footnotesize, % size of fonts used for the code
  breaklines, breakatwhitespace,     % automatic line breaking only at whitespace
  commentstyle=\color{mygreen},      % comment style
  numbers=left,
  showstringspaces=false,
  numberstyle=\tiny,
  frame=l,
  escapeinside={*@}{@*},           % if you want to add LaTeX within your code
  keywordstyle=\color{blue},         % keyword style
  stringstyle=\color{mymauve}        % string literal style
}

% ------------------------

\AtBeginDocument{%
        \renewcommand{\tablename}{Tab.}
        \renewcommand{\figurename}{Rys.}
}

% ------------------------
% --- < tabele > ---

\usepackage{array}
\usepackage{tabularx}
\usepackage{multirow}
\usepackage{booktabs}
\usepackage{makecell}
\usepackage[flushleft]{threeparttable}

% defines the X column to use m (\parbox[c]) instead of p (`parbox[t]`)
\newcolumntype{C}[1]{>{\hsize=#1\hsize\centering\arraybackslash}X}


%---------------------------------------------------------------------------

\author{{[Imiona i Nazwisko]}}

\makeatletter% Poniższe makra są wyłącznie zdefiniowane w klasie aghdpl-imir
\@ifclassloaded{aghdpl}{%

  \sex{m} % Rodzaj osobowych form czasowników: m - męski, ż - żeński
  \shortauthor{{[Im.\ i nazwisko]}}
  \albumnum{{[xxxxxx]}}
  \address{{[Adres]}}

  \titlePL{{[Bardzo długi temat niezwykle dogłębnej pracy dyplomowej inżynierskiej debatującej nad ekstraordynaryjnie pasjonującym zagadnieniem]}}
  \titleEN{{[Very long title of extremely indepth engineer dimploma thesis exploring an extraordinarily interesing subject]}}

  \shorttitlePL{{[Skrócony temat pracy]}} % skrócona wersja tytułu
  \shorttitleEN{{[Short thesis subject]}}

  % rodzaj pracy bez końcówki fleksyjnej np. inżyniersk, magistersk
  \thesistypePL{magistersk}
  \thesistypeEN{engineer}

  \supervisor{{[Tytuł, imię i nazwisko promotora]}}

  \reviewer{{[Tytuł, imię i nazwisko recenzenta]}}

  \degreeprogrammePL{{[Nazwa kierunku studiów]}}
  \degreeprogrammeEN{{[Field of Study]}}

  \specialisationPL{{[Nazwa specjalności]}}
  \specialisationEN{{[Specialisation]}}

  \graduationyear{{[Rok ukończenia]}}
  \years{2017/2018}
  \yearofstudy{IV}
  \formPL{stacjonarne}
  \formEN{full-time}

  % zgoda na publikację pracy w internecie: t-zgoda, cokolwiek 
  % innego-brak zgody
  \agree{t}

  % praktyka (dyplomowa)
  \apprenticeship{{[Praktyka dyplomowa]}}

  \department{Katedra Transportu Linowego}

  \facultyPL{Wydział Inżynierii Mechanicznej i Robotyki}
  \facultyEN{Faculty of Mechanical Engineering and Robotics}

  \thesisplan{% Przykładowy plan pracy, należy omówić z promotorem
    \begin{enumerate}
    \item Omówienie tematu pracy i sposobu realizacji z promotorem.
    \item Zebranie i opracowanie literatury dotyczącej tematu pracy.
    \item Zebranie i opracowanie wyników badań.
    \item Analiza wyników badań, ich omówienie i zatwierdzenie przez promotora.
    \item Opracowanie redakcyjne.
    \end{enumerate}
  }

  \summaryPL{\indent\indent%
	  {[Treść streszczenia]}
  }
  \summaryEN{\indent\indent%
	  {[Summary text]}
  }

  \acknowledgements{%
    Serdecznie dziękuję \dots tu ciąg dalszych podziękowań np.\ dla promotora,
    żony, sąsiada itp.
  }

  \setlength{\cftsecnumwidth}{10mm}
}{}%
\makeatother%

\date{\today}

%---------------------------------------------------------------------------
\setcounter{secnumdepth}{4}
\brokenpenalty=10000\relax

\begin{document}

\titlepages{}

% Ponowne zdefiniowanie stylu `plain`, aby usunąć numer strony z pierwszej strony spisu treści i poszczególnych rozdziałów.
\fancypagestyle{plain}
{%
        % Usuń nagłówek i stopkę
        \fancyhf{}
        % Usuń linie.
        \renewcommand{\headrulewidth}{0pt}
        \renewcommand{\footrulewidth}{0pt}
}

\setcounter{tocdepth}{2}
{\singlespacing\tableofcontents}
\clearpage

\include{rozdzial3}
\include{rozdzial1}
\include{rozdzial2}
\include{tests}

%Kod poniżej dodaje Bibliografię do spisu treści
\cleardoublepage{}
\phantomsection{}
\addcontentsline{toc}{chapter}{Bibliografia}
\printbibliography{}

% Załączniki
\begin{appendices}
  \makeatletter% Kod poniżej powoduje ustęp w spisie treści
  \addtocontents{toc}{\let\protect\l@chapter\protect\l@section}
  \chapter{Lorem Ipsum}

Lorem ipsum dolor sit amet, consectetur adipiscing elit. Fusce aliquet consequat sollicitudin. Nam nec eros ut dolor vulputate maximus. Vivamus quis neque sed orci cursus ornare. Proin vel elit eros. Duis efficitur mi tempus mi volutpat ullamcorper. Vestibulum consectetur dictum dui, ac suscipit eros aliquet ac. Quisque at dignissim mauris. Nulla non finibus nunc. In hac habitasse platea dictumst. Donec semper in nunc eget ultricies. Fusce varius scelerisque cursus. Vestibulum a sem lobortis, pretium nibh quis, pharetra justo.

Mauris turpis nunc, dignissim ac fringilla quis, dignissim sed dui. Cras porttitor congue nulla, vitae hendrerit ligula hendrerit vel. Donec lorem lectus, tempor a feugiat et, ultrices at augue. Suspendisse ultricies massa vitae pellentesque accumsan. Phasellus sollicitudin hendrerit lorem, lobortis aliquet nibh tristique a. Etiam nec tempus enim. Aenean diam nibh, pretium tincidunt malesuada vitae, laoreet non orci. Fusce dictum feugiat eros in malesuada. Sed vel ligula nunc. Donec nec hendrerit mauris. Sed accumsan quis quam vitae interdum. Praesent nec arcu est. Donec cursus nisi vitae ligula pharetra, quis sagittis felis dignissim.

Nunc a dapibus elit, nec iaculis erat. Suspendisse eleifend neque ac odio volutpat vulputate. Etiam varius odio quis leo aliquam, ac laoreet turpis vulputate. Donec rutrum pulvinar odio, vitae maximus ipsum bibendum a. Curabitur euismod erat a cursus vehicula. Aenean quis quam vulputate, consequat erat vitae, molestie felis. Morbi gravida nulla vitae leo hendrerit iaculis. Donec lobortis, quam ac ultrices aliquam, purus lectus tempus enim, et dictum nunc tortor vitae ligula. Aliquam volutpat bibendum nulla, non elementum nunc congue ut. Aenean mattis arcu in velit faucibus, vel eleifend nibh imperdiet. Etiam eget gravida nulla, quis varius sapien.

Donec bibendum commodo mi, vel interdum neque mattis nec. Suspendisse potenti. In lectus elit, accumsan non purus quis, viverra eleifend ipsum. Sed vel ex sed dolor consectetur scelerisque. Praesent id leo ultrices eros ornare accumsan. Mauris pharetra justo at tortor scelerisque mattis. Quisque eu risus vitae mi tristique pulvinar eget et nisi. Praesent a nibh vitae nunc aliquet laoreet eu eu massa. Sed aliquet mollis cursus. In erat nisl, suscipit eget justo id, euismod pellentesque nulla.

  % \input{zalocznikB}
  % itd.
  \makeatother%
\end{appendices}

\end{document}
