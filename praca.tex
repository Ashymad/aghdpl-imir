\documentclass[pl,12pt]{aghdpl}
% \documentclass[en,11pt]{aghdpl}  % praca w języku angielskim

% Lista wszystkich języków stanowiących języki pozycji bibliograficznych użytych w pracy.
% (Zgodnie z zasadami tworzenia bibliografii każda pozycja powinna zostać utworzona zgodnie z zasadami języka, w którym dana publikacja została napisana.)
\usepackage[english,polish]{babel}

% Użyj polskiego łamania wyrazów (zamiast domyślnego angielskiego).
\usepackage{polski}

\usepackage[utf8]{inputenc}

% Załączniki

\usepackage[toc, page]{appendix}
\renewcommand\appendixpagename{Załączniki}
\renewcommand\appendixtocname{Załączniki}

% dodatkowe pakiety

\usepackage{mathtools}
\usepackage{amsfonts}
\usepackage{amsmath}
\usepackage{amsthm}
\usepackage{float}% do umieszczenia floatów [H]
\usepackage{enumitem}
\setlist{nosep} % or \setlist{noitemsep} to leave space around whole list
\usepackage[bookmarks,hidelinks]{hyperref}

% Środowisko float do kodu źródłowego \begin{program}

\floatstyle{plaintop}
\ifcsname{chapter}\endcsname%
    \newfloat{program}{!tbh}{lop}[chapter]
\else%
    \newfloat{program}{!tbh}{lop}
\fi
\floatname{program}{Kod źr.}

% Kod poniżej powoduje, że floaty nie wylatują poza granice sekcji

\usepackage{placeins}

\ifcsname{chapter}\endcsname%
    \let\Oldchapter\chapter
    \renewcommand{\chapter}{\FloatBarrier\Oldchapter}
\fi

\let\Oldsection\section
\renewcommand{\section}{\FloatBarrier\Oldsection}

\let\Oldsubsection\subsection
\renewcommand{\subsection}{\FloatBarrier\Oldsubsection}

\let\Oldsubsubsection\subsubsection
\renewcommand{\subsubsection}{\FloatBarrier\Oldsubsubsection}

% --- < bibliografia > ---


\usepackage[
style=numeric,
sorting=none,
%
% Zastosuj styl wpisu bibliograficznego właściwy językowi publikacji.
language=autobib,
autolang=other,
% Zapisuj datę dostępu do strony WWW w formacie RRRR-MM-DD.
urldate=edtf,
seconds=true,
% Nie dodawaj numerów stron, na których występuje cytowanie.
backref=false,
% Podawaj ISBN.
isbn=true,
% Nie podawaj URL-i, o ile nie jest to konieczne.
url=false,
%
% Ustawienia związane z polskimi normami dla bibliografii.
maxbibnames=3,
% Jeżeli używamy Bibera:
backend=biber
]{biblatex}

\usepackage{csquotes}
% Ponieważ `csquotes` nie posiada polskiego stylu, można skorzystać z mocno zbliżonego stylu chorwackiego.
\DeclareQuoteAlias{croatian}{polish}

\addbibresource{bibliografia.bib}

% Nie wyświetlaj wybranych pól.
%\AtEveryBibitem{\clearfield{note}}


% ------------------------
% --- < listingi > ---

% Użyj czcionki kroju Times.
\usepackage{newtxtext}
\usepackage{newtxmath}

\usepackage{listings}
\lstloadlanguages{TeX}

\lstset{
        literate={ą}{{\k{a}}}1
           {ć}{{\'c}}1
           {ę}{{\k{e}}}1
           {ó}{{\'o}}1
           {ń}{{\'n}}1
           {ł}{{\l{}}}1
           {ś}{{\'s}}1
           {ź}{{\'z}}1
           {ż}{{\.z}}1
           {Ą}{{\k{A}}}1
           {Ć}{{\'C}}1
           {Ę}{{\k{E}}}1
           {Ó}{{\'O}}1
           {Ń}{{\'N}}1
           {Ł}{{\L{}}}1
           {Ś}{{\'S}}1
           {Ź}{{\'Z}}1
           {Ż}{{\.Z}}1
}

% Ustawienia pakietu lstlisting do umieszczania kodu

\usepackage{color}

\definecolor{mygreen}{rgb}{0,0.6,0}
\definecolor{mygray}{rgb}{0.5,0.5,0.5}
\definecolor{mymauve}{rgb}{0.58,0,0.82}

\lstset{ %
  backgroundcolor=\color{white},     % choose the background color
  basicstyle=\ttfamily\footnotesize, % size of fonts used for the code
  breaklines, breakatwhitespace,     % automatic line breaking only at whitespace
  commentstyle=\color{mygreen},      % comment style
  numbers=left,
  showstringspaces=false,
  numberstyle=\tiny,
  frame=l,
  escapeinside={\%*}{*)},            % if you want to add LaTeX within your code
  keywordstyle=\color{blue},         % keyword style
  stringstyle=\color{mymauve}        % string literal style
}

% ------------------------

\AtBeginDocument{
        \renewcommand{\tablename}{Tab.}
        \renewcommand{\figurename}{Rys.}
}

% ------------------------
% --- < tabele > ---

\usepackage{array}
\usepackage{tabularx}
\usepackage{multirow}
\usepackage{booktabs}
\usepackage{makecell}
\usepackage[flushleft]{threeparttable}

% defines the X column to use m (\parbox[c]) instead of p (`parbox[t]`)
\newcolumntype{C}[1]{>{\hsize=#1\hsize\centering\arraybackslash}X}


%---------------------------------------------------------------------------

\author{{[}Imię i Nazwisko]}

\makeatletter% Poniższe makra są wyłącznie zdefiniowane w klasie aghdpl-imir
\@ifclassloaded{aghdpl}{

  \sex{m} % Mężczyzna - m; kobieta - cokolwiek
  \shortauthor{{[}Im. i nazwisko]}
  \albumnum{{[}xxxxxx]}
  \address{{[}Adres]}

  \titlePL{{[}Bardzo długi temat niezwykle dogłębnej pracy dyplomowej inżynierskiej debatującej nad ekstraordynaryjnie pasjonującym zagadnieniem]}
  \titleEN{{[}Very long title of extremely indepth engineer dimploma thesis exploring an extraordinarily interesing subject]}

  \shorttitlePL{{[}Skrócony temat pracy]} % skrócona wersja tytułu
  \shorttitleEN{{[}Short thesis subject]}

  % rodzaj pracy bez końcówki fleksyjnej np. inżyniersk, magistersk
  \thesistypePL{inżyniersk}
  \thesistypeEN{engineer}

  \supervisor{{[}Tytuł, imię i nazwisko promotora]}

  \reviewer{{[}Tytuł, imię i nazwisko recenzenta]}

  \degreeprogrammePL{{[}Nazwa kierunku studiów]}
  \degreeprogrammeEN{{[}Field of Study]}

  \specialisationPL{{[}Nazwa specjalności]}
  \specialisationEN{{[}Specialisation]}

  \graduationyear{{[}Rok ukończenia]}
  \years{2017/2018}
  \yearofstudy{IV}

  \department{Katedra Transportu Linowego}

  \facultyPL{Wydział Inżynierii Mechanicznej i Robotyki}
  \facultyEN{Faculty of Mechanical Engineering and Robotics}

  \thesisplan{ % Przykłądowy plan pracy, należy omówić z promotorem
    \begin{enumerate}
    \item Omówienie tematu pracy i sposobu realizacji z promotorem.
    \item Zebranie i opracowanie literatury dotyczącej tematu pracy.
    \item Zebranie i opracowanie wyników badań.
    \item Analiza wyników badań, ich omówienie i zatwierdzenie przez promotora.
    \item Opracowanie redakcyjne.
    \end{enumerate}
  }

  \summaryPL{\indent\indent%
    {[}Treść streszczenia]
  }
  \summaryEN{\indent\indent%
    {[}Summary text]
  }

  \acknowledgements{%
    Serdecznie dziękuję \dots tu ciąg dalszych podziękowań np. dla promotora,
    żony, sąsiada itp.
  }

  \setlength{\cftsecnumwidth}{10mm}
}{}%
\makeatother%

\date{\today}

%---------------------------------------------------------------------------
\setcounter{secnumdepth}{4}
\brokenpenalty=10000\relax

\begin{document}

\titlepages

% Ponowne zdefiniowanie stylu `plain`, aby usunąć numer strony z pierwszej strony spisu treści i poszczególnych rozdziałów.
\fancypagestyle{plain}
{
        % Usuń nagłówek i stopkę
        \fancyhf{}
        % Usuń linie.
        \renewcommand{\headrulewidth}{0pt}
        \renewcommand{\footrulewidth}{0pt}
}

\setcounter{tocdepth}{2}
{\singlespacing\tableofcontents}
\clearpage

\chapter{Przykłady elementów pracy dyplomowej}

\section{Liczba}

Pakiet \texttt{siunitx} zadba o to, by liczba została poprawnie sformatowana: \\
\begin{center}
        \num{1234567890.0987654321}
\end{center}


\section{Rysunek}

Pakiet \texttt{subcaption} pozwala na umieszczanie w podpisie rysunku odnośników do ,,podilustracji'':\\ % chktex 26

\begin{figure}[h]
        \centering
        \begin{subfigure}{0.35\textwidth}
                \centering
                \framebox[2.0\width]{A}
                \subcaption{\label{subfigure_a}}
        \end{subfigure}
        \begin{subfigure}{0.35\textwidth}
                \centering
                \framebox[2.0\width]{B}
                \subcaption{\label{subfigure_b}}
        \end{subfigure}\label{fig:subcaption_example}
	\caption{Przykład użycia \texttt{\textbackslash{} subcaption}: \protect\subref{subfigure_a} litera A, \protect\subref{subfigure_b} litera B.}
\end{figure}

\section{Tabela}

Pakiet \texttt{threeparttable} umożliwia dodanie do tabeli adnotacji: \\

\begin{table}[h]
        \centering
        
        \begin{threeparttable}
                \caption{Przykład tabeli}\label{tab:table_example}
                
                \begin{tabularx}{0.6\textwidth}{C{1}}
                        \toprule
                        \thead{Nagłówek\tnote{a}} \\
                        \midrule
                        Tekst 1 \\
                        Tekst 2 \\
                        \bottomrule
                \end{tabularx}
                
                \begin{tablenotes}
                        \footnotesize
			\item[a] Jakiś komentarz\textellipsis{}
                \end{tablenotes}
                
        \end{threeparttable}
\end{table}

\section{Wzory matematyczne}

Czasem zachodzi potrzeba wytłumaczenia znaczenia symboli użytych w równaniu. Można to zrobić z użyciem zdefiniowanego na potrzeby niniejszej klasy środowiska \texttt{eqwhere}.

\begin{equation}
E = mc^2
\end{equation}
gdzie
\begin{eqwhere}[2cm]
        \item[$m$] masa
        \item[$c$] prędkość światła w próżni
\end{eqwhere}

Odległość półpauzy od lewego marginesu należy dobrać pod kątem najdłuższego symbolu (bądź listy symboli) poprzez odpowiednie ustawienie parametru tego środowiska (domyślnie: 2 cm).

\chapter{Wprowadzenie}\label{cha:wprowadzenie}

\LaTeX~jest systemem składu umożliwiającym tworzenie dowolnego typu dokumentów (w~szczególności naukowych i~technicznych) o~wysokiej jakości typograficznej (\cite{Dil00}, \cite{Lam92}). Wysoka jakość składu jest niezależna od rozmiaru dokumentu --- zaczynając od krótkich listów do bardzo grubych książek. \LaTeX~automatyzuje wiele prac związanych ze składaniem dokumentów np.: referencje, cytowania, generowanie spisów (treśli, rysunków, symboli itp.) itd.

\LaTeX~jest zestawem instrukcji umożliwiających autorom skład i wydruk ich prac na najwyższym poziomie typograficznym. Do formatowania dokumentu \LaTeX~stosuje \TeX{}a (wymiawamy [tech] --- greckie litery $\tau$, $\epsilon$, $\chi$). Korzystając z~systemu składu \LaTeX~mamy za zadanie przygotować jedynie tekst źródłowy, cały ciężar składania, formatowania dokumentu przejmuje na siebie system.

%---------------------------------------------------------------------------

\section{Cele pracy}\label{sec:celePracy}


Celem poniższej pracy jest zapoznanie studentów z systemem \LaTeX~w zakresie umożliwiającym im samodzielne, profesjonalne złożenie pracy dyplomowej w systemie \LaTeX.

\subsection{Jakiś tytuł}

\subsubsection{Jakiś tytuł w subsubsection}


\subsection{Jakiś tytuł 2}

%---------------------------------------------------------------------------

\section{Zawartość pracy}\label{sec:zawartoscPracy}

W rodziale~\ref{cha:pierwszyDokument} przedstawiono podstawowe informacje dotyczące struktury dokumentów w \LaTeX{}u. Alvis~\cite{Alvis2011} jest językiem 



















\chapter{Pierwszy dokument}\label{cha:pierwszyDokument}

W rozdziale tym przedstawiono podstawowe informacje dotyczące struktury prostych plików \LaTeX{}a. Omówiono również metody kompilacji plików z zastosowaniem programów \emph{latex} oraz \emph{pdflatex}.

%---------------------------------------------------------------------------

\section{Struktura dokumentu}\label{sec:strukturaDokumentu}

Plik \LaTeX{}owy jest plikiem tekstowym, który oprócz tekstu zawiera polecenia formatujące ten tekst (analogicznie do języka HTML). Plik składa się z dwóch części:
\begin{enumerate}%[1)]
\item Preambuły --- określającej klasę dokumentu oraz zawierającej m.in.\ polecenia dołączającej dodatkowe pakiety;

\item Części głównej --- zawierającej zasadniczą treść dokumentu.
\end{enumerate}

\begin{program}
  \caption{Część głowna}
\begin{lstlisting}
\documentclass[a4paper,12pt]{article}      % preambuła
\usepackage[polish]{babel}
\usepackage[utf8]{inputenc}
\usepackage[T1]{fontenc}
\usepackage{times}

\begin{document}                           % część główna

\section{Sztuczne życie}

% treść
% ąśężźćńłóĘŚĄŻŹĆŃÓŁ

\end{document}
\end{lstlisting}
\end{program}

Nie ma żadnych przeciwskazań do tworzenia dokumentów w~\LaTeX u w~języku polskim. Plik źródłowy jest zwykłym plikiem tekstowym i~do jego przygotowania można użyć dowolnego edytora tekstów, a~polskie znaki wprowadzać używając prawego klawisza \texttt{Alt}. Jeżeli po kompilacji dokumentu polskie znaki nie są wyświetlane poprawnie, to na 95\% źle określono sposób kodowania znaków (należy zmienić opcje wykorzystywanych pakietów).


%---------------------------------------------------------------------------

\section{Kompilacja}
\label{sec:kompilacja}


Załóżmy, że przygotowany przez nas dokument zapisany jest w pliku \texttt{test.tex}. Kolejno wykonane poniższe polecenia (pod warunkiem, że w pierwszym przypadku nie wykryto błędów i kompilacja zakończyła się sukcesem) pozwalają uzyskać nasz dokument w formacie pdf:
\begin{lstlisting}
latex test.tex
dvips test.dvi -o test.ps
ps2pdf test.ps
\end{lstlisting}
%
lub za pomocą PDF\LaTeX:
\begin{lstlisting}
pdflatex test.tex
\end{lstlisting}

Przy pierwszej kompilacji po zmiane tekstu, dodaniu nowych etykiet itp., \LaTeX~tworzy sobie spis rozdziałów, obrazków, tabel itp., a dopiero przy następnej kompilacji korzysta z tych informacji.

W pierwszym przypadku rysunki powinny być przygotowane w~formacie eps, a~w~drugim w~formacie pdf. Ponadto, jeżeli używamy polecenia \texttt{pdflatex test.tex} można wstawiać grafikę bitową (np. w formacie jpg).



%---------------------------------------------------------------------------

\section{Narzędzia}
\label{sec:narzedzia}


Do przygotowania pliku źródłowego może zostać wykorzystany dowolny edytor tekstowy. Niektóre edytory, np. GEdit, mają wbudowane moduły ułatwiające składanie tekstów w LaTeXu (kolorowanie składni, skrypty kompilacji, itp.).

Jednym z bardziej znanych środowisk do składania dokumentów  \LaTeX a jest {\em TeXstudio}, oferujące kompletne środowisko pracy. Zobacz: \url{http://www.texstudio.org}


Bardzo dobrym środowiskiem jest również edytor gEdit z wtyczką obsługującą \LaTeX a. Jest to standardowy edytor środowiska Gnome. Po instalacji wtyczki obsługującej \LaTeX~ zamienia się w wygodne i szybkie środowisko pracy.

\textbf{Dla testu łamania stron powtórzenia powyższego tekstu.}


Do przygotowania pliku źródłowego może zostać wykorzystany dowolny edytor tekstowy. Niektóre edytory, np. GEdit, mają wbudowane moduły ułatwiające składanie tekstów w LaTeXu (kolorowanie składni, skrypty kompilacji, itp.).
Jednym z bardziej znanych środowisk do składania dokumentów  \LaTeX a jest {\em TeXstudio}, oferujące kompletne środowisko pracy. Zobacz: \url{http://www.texstudio.org}
Bardzo dobrym środowiskiem jest również edytor gEdit z wtyczką obsługującą \LaTeX a. Jest to standardowy edytor środowiska Gnome. Po instalacji wtyczki obsługującej \LaTeX~ zamienia się w wygodne i szybkie środowisko pracy.
Po instalacji wtyczki obsługującej \LaTeX~ zamienia się w wygodne i szybkie środowisko pracy.

Do przygotowania pliku źródłowego może zostać wykorzystany dowolny edytor tekstowy. Niektóre edytory, np. GEdit, mają wbudowane moduły ułatwiające składanie tekstów w LaTeXu (kolorowanie składni, skrypty kompilacji, itp. itd. itp.).
Jednym z bardziej znanych środowisk do składania dokumentów  \LaTeX a jest {\em TeXstudio}, oferujące kompletne środowisko pracy. Zobacz: \url{http://www.texstudio.org}
Bardzo dobrym środowiskiem jest również edytor gEdit z wtyczką obsługującą \LaTeX a. Jest to standardowy edytor środowiska Gnome. Po instalacji wtyczki obsługującej \LaTeX~ zamienia się w wygodne i szybkie środowisko pracy.

Do przygotowania pliku źródłowego może zostać wykorzystany dowolny edytor tekstowy. Niektóre edytory, np. GEdit, mają wbudowane moduły ułatwiające składanie tekstów w LaTeXu (kolorowanie składni, skrypty kompilacji, itp.).
Jednym z bardziej znanych środowisk do składania dokumentów  \LaTeX a jest {\em TeXstudio}, oferujące kompletne środowisko pracy. Zobacz: \url{http://www.texstudio.org}
Bardzo dobrym środowiskiem jest również edytor gEdit z wtyczką obsługującą \LaTeX a. Jest to standardowy edytor środowiska Gnome. Po instalacji wtyczki obsługującej \LaTeX~ zamienia się w wygodne i szybkie środowisko pracy.

%---------------------------------------------------------------------------

\section{Przygotowanie dokumentu}
\label{sec:przygotowanieDokumentu}

Plik źródłowy \LaTeX a jest zwykłym plikiem tekstowym. Przygotowując plik
źródłowy warto wiedzieć o kilku szczegółach:

\begin{itemize}
\item
Poszczególne słowa oddzielamy spacjami, przy czym ilość spacji nie ma znaczenia.
Po kompilacji wielokrotne spacje i tak będą wyglądały jak pojedyncza spacja.
Aby uzyskać {\em twardą spację}, zamiast znaku spacji należy użyć znaku {\em
tyldy}.

\item
Znakiem końca akapitu jest pusta linia (ilość pusty linii nie ma znaczenia), a
nie znaki przejścia do nowej linii.

\item
\LaTeX~sam formatuje tekst. \textbf{Nie starajmy się go poprawiać}, chyba, że
naprawdę wiemy co robimy.
\end{itemize} 



\chapter{Testy}

\section{Test URL-a}

Wejdź na stronę \url{https://www.google.pl/} i wpisz szukane zdanie.

\clearpage

\section{Test dzielenia wdów}

Lorem ipsum dolor sit amet, ex est alia dolorem commune. Duo modo errem no. Ea harum doming atomorum mei. Consul animal malorum cu qui, sumo dicta graece an est, vim ei clita regione.

Vel eu quando doming fastidii, mei graeco indoctum an, legere theophrastus in pro. Te mei probatus eleifend interpretaris. Est no autem liber vituperatoribus, cu mea dicam constituto. Ea laudem tritani consectetuer sit, sanctus patrioque expetendis vix in. Duo id fugit adversarium signiferumque, an quot modus molestiae qui.

Ut paulo definiebas pro. Mea an quod esse. Et atomorum facilisis moderatius sit. Graeco iudicabit forensibus in vel. Eam cu lorem aeterno offendit, cu vix nulla congue posidonium. Vel lucilius evertitur vituperata no.

Mea eu graecis prodesset. Et tota eius nec. Ei etiam oratio has, vel ei homero eripuit invenire. Sed ex errem intellegebat, sea et elitr intellegat constituto. Nostro voluptua accusamus eos in, ei sale admodum has. Vim ne consetetur reformidans, ad has malis recusabo persequeris, per etiam virtute invenire in.

Te nihil eruditi eam, sit aperiam accusam mediocritatem at. Nec ne nonumy dictas disputationi, vis ridens sadipscing ex. Harum euripidis ex vix, at consetetur instructior signiferumque mel, at mei elitr honestatis. Id sit congue vituperata. Temporibus eloquentiam no eum.

Pro id esse phaedrum, nostro iudicabit eos ut. Sit ea aperiam alienum, harum audiam voluptua cu usu. Iudico invenire te vel, id suscipit disputando pri. Ut sumo expetenda mea.

Cum at idque nullam aperiam, vis ex aeque ponderum luptatum. Vix soluta graeco dissentiet ut, ut est reque periculis similique, ut dicta dicant repudiare sea. Ne dolor legendos signiferumque ius, at eirmod convenire qui. Suas numquam conceptam mei ex. Autem homero eos et, sea dicta alienum iudicabit ut.

Ea duo consulatu vulputate, id elit perpetua cum. His ei aeque saepe audiam. Prompta laoreet facilisi ne sed, per hinc consetetur te, oratio fuisset ullamcorper mel at. Quis suscipiantur ne nec, agam efficiendi usu in.

Vis eu iuvaret singulis appellantur, usu ex saepe omittantur. Sed possit mnesarchum at, usu illum choro oratio in, et debet dolor vix. Mel aperiri suscipiantur ne, te per illum fuisset, lorem pericula mei ad. Pri id tale lucilius dissentiet, id sea sonet expetenda. Agam sensibus persequeris sed no, eum at tamquam sanctus.

Omnis exerci soleat ut vis. Rebum vidisse sea ex. Ius animal gubergren efficiantur ad, mollis probatus nec ut. Meis platonem ex vel, ut qui tale tritani equidem. Vide meis fuisset mel at, nam an assum delenit gubergren. No illum reprimique vim, te augue nullam per, ludus dicant suscipiantur ne sed.

An pri mediocrem deseruisse, ad sumo audire dissentiet sit. Sit ea civibus lobortis. Etiam ceteros commune ei vis. Pro ei equidem vivendo. Quo ne prima periculis omittantur, ex rebum veritus sit, ei dolor maiestatis mea.

\subsection{Lorem ipsum}

Et mel munere quodsi sapientem. Essent legimus ne pro. Est ornatus definiebas et. No habemus docendi ius, purto sapientem mei at. Tamquam vivendo necessitatibus has at, no habemus praesent nec. No quo modus iudicabit scriptorem. Modus intellegebat ea vim. Cu ius lorem regione offendit, ne accusata sensibus vituperatoribus quo. Sit ut iuvaret indoctum. Ut mea sale justo. Sapientem definitionem ius eu, at sea quem doming. Facete conclusionemque ut nec, vix at duis eius. Eos quot consequuntur et, ornatus liberavisse ne mei.

Per an dicam commodo tractatos, usu in timeam numquam tacimates. Case delectus eu sea, usu audiam eleifend tincidunt id, nec at decore discere mentitum. Ut elit veri eloquentiam his, ceteros tractatos ea has. Duo impetus scribentur et, eu quo errem everti, ad recusabo consulatu ius. Fastidii comprehensam pri ea, ex duo augue quando denique. Eos aeterno deserunt sententiae cu, ius quas tation patrioque ex.

Id autem scripta explicari nec, congue quidam possit te sit. Et usu ipsum bonorum graecis, ferri verear deterruisset eum cu. Purto porro accommodare cu vim. Cum ei tritani pertinacia voluptaria.

%Kod poniżej dodaje Bibliografię do spisu treści
\cleardoublepage
\phantomsection
\addcontentsline{toc}{chapter}{Bibliografia}
\printbibliography

% Załączniki
\begin{appendices}
  \makeatletter% Kod poniżej powoduje ustęp w spisie treści
  \addtocontents{toc}{\let\protect\l@chapter\protect\l@section}
  \chapter{Lorem Ipsum}

Lorem ipsum dolor sit amet, consectetur adipiscing elit. Fusce aliquet consequat sollicitudin. Nam nec eros ut dolor vulputate maximus. Vivamus quis neque sed orci cursus ornare. Proin vel elit eros. Duis efficitur mi tempus mi volutpat ullamcorper. Vestibulum consectetur dictum dui, ac suscipit eros aliquet ac. Quisque at dignissim mauris. Nulla non finibus nunc. In hac habitasse platea dictumst. Donec semper in nunc eget ultricies. Fusce varius scelerisque cursus. Vestibulum a sem lobortis, pretium nibh quis, pharetra justo.

Mauris turpis nunc, dignissim ac fringilla quis, dignissim sed dui. Cras porttitor congue nulla, vitae hendrerit ligula hendrerit vel. Donec lorem lectus, tempor a feugiat et, ultrices at augue. Suspendisse ultricies massa vitae pellentesque accumsan. Phasellus sollicitudin hendrerit lorem, lobortis aliquet nibh tristique a. Etiam nec tempus enim. Aenean diam nibh, pretium tincidunt malesuada vitae, laoreet non orci. Fusce dictum feugiat eros in malesuada. Sed vel ligula nunc. Donec nec hendrerit mauris. Sed accumsan quis quam vitae interdum. Praesent nec arcu est. Donec cursus nisi vitae ligula pharetra, quis sagittis felis dignissim.

Nunc a dapibus elit, nec iaculis erat. Suspendisse eleifend neque ac odio volutpat vulputate. Etiam varius odio quis leo aliquam, ac laoreet turpis vulputate. Donec rutrum pulvinar odio, vitae maximus ipsum bibendum a. Curabitur euismod erat a cursus vehicula. Aenean quis quam vulputate, consequat erat vitae, molestie felis. Morbi gravida nulla vitae leo hendrerit iaculis. Donec lobortis, quam ac ultrices aliquam, purus lectus tempus enim, et dictum nunc tortor vitae ligula. Aliquam volutpat bibendum nulla, non elementum nunc congue ut. Aenean mattis arcu in velit faucibus, vel eleifend nibh imperdiet. Etiam eget gravida nulla, quis varius sapien.

Donec bibendum commodo mi, vel interdum neque mattis nec. Suspendisse potenti. In lectus elit, accumsan non purus quis, viverra eleifend ipsum. Sed vel ex sed dolor consectetur scelerisque. Praesent id leo ultrices eros ornare accumsan. Mauris pharetra justo at tortor scelerisque mattis. Quisque eu risus vitae mi tristique pulvinar eget et nisi. Praesent a nibh vitae nunc aliquet laoreet eu eu massa. Sed aliquet mollis cursus. In erat nisl, suscipit eget justo id, euismod pellentesque nulla.
  % \input{zalocznikB}
  % itd.
  \makeatother%
\end{appendices}

\end{document}
